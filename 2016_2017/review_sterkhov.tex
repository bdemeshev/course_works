\documentclass[a4paper, 12pt]{article}


\usepackage{mathrsfs}

\usepackage{amscd}
\usepackage[paper=a4paper, top=20.0mm, bottom=20.0mm, left=20.0mm, right=20.0mm,
includefoot]{geometry}

\usepackage{ragged2e} % центрирование текста

\usepackage{tikz} % картинки в tikz
\usepackage{microtype} % свешивание пунктуации

\usepackage{floatrow} % для выравнивания рисунка и подписи
\usepackage{caption} % для пустых подписей

\usepackage{array} % для столбцов фиксированной ширины

\usepackage{indentfirst} % отступ в первом параграфе

\usepackage{sectsty} % для центрирования названий частей
\allsectionsfont{\centering}

\usepackage{amsmath, amsfonts} % куча стандартных математических плюшек

\usepackage{comment} % для комментариев

\usepackage{multicol} % текст в несколько колонок

\usepackage{lastpage} % чтобы узнать номер последней страницы

\usepackage{enumitem} % дополнительные плюшки для списков
%  например \begin{enumerate}[resume] позволяет продолжить нумерацию в новом списке

\usepackage{url} % для вставки интернет-ссылок

\usepackage{fontspec}
\usepackage{polyglossia}

\setmainlanguage{russian}
\setotherlanguages{english}

% download "Linux Libertine" fonts:
% http://www.linuxlibertine.org/index.php?id=91&L=1
\setmainfont{Linux Libertine O} % or Helvetica, Arial, Cambria
% why do we need \newfontfamily:
% http://tex.stackexchange.com/questions/91507/
\newfontfamily{\cyrillicfonttt}{Linux Libertine O}

\AddEnumerateCounter{\asbuk}{\russian@alph}{щ} % для списков с русскими буквами
\setlist[enumerate, 2]{label=\asbuk*),ref=\asbuk*}

\DeclareMathOperator{\Var}{Var}
\DeclareMathOperator{\E}{\mathbb{E}}

\let\P\relax
\DeclareMathOperator{\P}{\mathbb{P}}
\def\cN{\mathcal{N}}

\usepackage{fancyhdr} % весёлые колонтитулы
\pagestyle{fancy}
\lhead{}
\chead{}
\rhead{}
\lfoot{}
\cfoot{}
\rfoot{}
\renewcommand{\headrulewidth}{0.4pt}
\renewcommand{\footrulewidth}{0.4pt}

\providecommand{\tightlist}{%
  \setlength{\itemsep}{0pt}\setlength{\parskip}{0pt}}

\begin{document}


\begin{center}
{\small Федеральное государственное автономное образовательное учреждение\\ 
высшего профессионального образования «Национальный исследовательский\\ 
университет «Высшая школа экономики».}
\end{center}

\vspace{0.4cm}

\begin{center}
\textbf{Отзыв на выпускную квалификационную работу}
\end{center}

\vspace{0.4cm}

Студент ОП «Экономика»: Стерхов Сергей Игоревич

\vspace{0.4cm}

Научный руководитель: Демешев Борис Борисович

\vspace{0.4cm}

Тема: Построение модели прогноза LTV для условно бесплатного игрового проекта

\vspace{0.4cm}

Работа Сергея посвящена решению практической задачи. Основная задача ---
спрогнозировать величину LTV, сколько денег потратят пользователи в
игровом онлайн проекте. Работа построена на искусствено зашумленных
реальных данных.

В работе совсем нет графиков, хотя именно с этой задачи началось наше
обсуждение работы осенью. Это существенный минус.

Вторым существенным недостатком работы является обилие нереализованных
идей. Об этих идеях упоминает и сам Сергей. Можно было: отдельно
моделировать поведение игроков, тратящих крупные суммы в игре;
исследовать модель, прогнозирующую расходы не отдельных игроков, а
кластеров; учитывать признаки, связанные с поведением игрока в игре;
сравнить прогнозную силу исследуемого алгоритма с другими стандартными
алгоритмами прогнозирования LTV и так далее.

В работе есть и мелкие небрежности. Многие числа приводится с излишним
количеством знаков после запятой. Сначала концепция LTV излагается в
непрерывном времени, а затем происходит резкий переход в дискретное
время без комментариев.

Работа содержит небольшой но вполне релевантный обзор литературы о
моделировании LTV.

Большим плюсом ВКР является то, что Сергей решает реальную
бизнес-задачу. Много времени в таких задачах уходит на чистку данных.
Сергей подробно описывает, как проводится чистка данных.

Трудно оценивать работу студента не родного факультета, так как не с чем
сравнить. Если бы ВКР Сергея защищалась на факультете экономики, то в
зависимости от комиссии и презентации получила бы от удовлетворительно
до хорошо.

\vspace{0.4cm}

05 июня 2017

Демешев Борис Борисович



\end{document}

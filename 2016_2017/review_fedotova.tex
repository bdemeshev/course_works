\documentclass[a4paper, 12pt]{article}


\usepackage{mathrsfs}

\usepackage{amscd}
\usepackage[paper=a4paper, top=20.0mm, bottom=20.0mm, left=20.0mm, right=20.0mm,
includefoot]{geometry}

\usepackage{ragged2e} % центрирование текста

\usepackage{tikz} % картинки в tikz
\usepackage{microtype} % свешивание пунктуации

\usepackage{floatrow} % для выравнивания рисунка и подписи
\usepackage{caption} % для пустых подписей

\usepackage{array} % для столбцов фиксированной ширины

\usepackage{indentfirst} % отступ в первом параграфе

\usepackage{sectsty} % для центрирования названий частей
\allsectionsfont{\centering}

\usepackage{amsmath, amsfonts} % куча стандартных математических плюшек

\usepackage{comment} % для комментариев

\usepackage{multicol} % текст в несколько колонок

\usepackage{lastpage} % чтобы узнать номер последней страницы

\usepackage{enumitem} % дополнительные плюшки для списков
%  например \begin{enumerate}[resume] позволяет продолжить нумерацию в новом списке

\usepackage{url} % для вставки интернет-ссылок

\usepackage{fontspec}
\usepackage{polyglossia}

\setmainlanguage{russian}
\setotherlanguages{english}

% download "Linux Libertine" fonts:
% http://www.linuxlibertine.org/index.php?id=91&L=1
\setmainfont{Linux Libertine O} % or Helvetica, Arial, Cambria
% why do we need \newfontfamily:
% http://tex.stackexchange.com/questions/91507/
\newfontfamily{\cyrillicfonttt}{Linux Libertine O}

\AddEnumerateCounter{\asbuk}{\russian@alph}{щ} % для списков с русскими буквами
\setlist[enumerate, 2]{label=\asbuk*),ref=\asbuk*}

\DeclareMathOperator{\Var}{Var}
\DeclareMathOperator{\E}{\mathbb{E}}

\let\P\relax
\DeclareMathOperator{\P}{\mathbb{P}}
\def\cN{\mathcal{N}}

\usepackage{fancyhdr} % весёлые колонтитулы
\pagestyle{fancy}
\lhead{}
\chead{}
\rhead{}
\lfoot{}
\cfoot{}
\rfoot{}
\renewcommand{\headrulewidth}{0.4pt}
\renewcommand{\footrulewidth}{0.4pt}

\providecommand{\tightlist}{%
  \setlength{\itemsep}{0pt}\setlength{\parskip}{0pt}}

\begin{document}


\begin{center}
{\small Федеральное государственное автономное образовательное учреждение\\ 
высшего профессионального образования «Национальный исследовательский\\ 
университет «Высшая школа экономики».}
\end{center}

\vspace{0.4cm}

\begin{center}
\textbf{Отзыв на выпускную квалификационную работу}
\end{center}

\vspace{0.4cm}

Студент ОП «Экономика»: Федотова Мирослава Андреевна

\vspace{0.4cm}

Научный руководитель: Демешев Борис Борисович

\vspace{0.4cm}

Тема: Выявление индивидуальных предпочтений с помощью байесовской
иерархической мультиномиальной логит-модели

\vspace{1cm}

Работа Мирославы посвящена анализу предпочтений индивидов. Мирослава
анализирует данные реального эксперимента, в котором индивидам
предлагается выбор из нескольких альтернативных пакетов мобильной связи.

В своей работе Мирослава:

\begin{enumerate}
\def\labelenumi{\arabic{enumi}.}
\item
  Успешно формулирует и оценивает иерархическую байесовскую
  мультиномиальную модель с разным количеством кластеров.
\item
  Проверяет устойчивость выводов к изменениям априорного распределения.
\item
  Определяет оптимальное число кластеров абонентов для получения
  наилучших прогнозов.
\end{enumerate}

Мирослава находит, что оптимальное число кластеров достаточно велико,
доказывая тем самым преимущество байесовского подхода в данном
конкретном случае. Оценить модель с различными коэффициентами в каждом
из сотни кластеров в рамках классического максимального правдоподобия
было бы невозможно.

Достоинство модели со множеством кластеров состоит не только в более
точных прогнозах по сравнению с агрегированной моделью, но и в более
точном определении предпочтений каждого отдельного индивида.

Было бы очень здорово реализовать автоматическое определение количества
кластеров с помощью байесовского подхода. К сожалению, на это не хватило
времени.

Мирослава отлично разобралась байесовском подходе к оцениванию моделей,
сформулировала требуемую модель на языке STAN и визуализировала
результаты в среде R. Кроме иерархических мультиномиальных
логит-моделей, Мирослава использовала нелинейный алгоритм t-SNE для
снижения размерности и кластеризации абонентов.

Работа Мирославы заслуживает отличной оценки.

\vspace{1cm}

15 мая 2017

Демешев Борис Борисович



\end{document}

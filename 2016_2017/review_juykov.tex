\documentclass[a4paper, 12pt]{article}


\usepackage{mathrsfs}

\usepackage{amscd}
\usepackage[paper=a4paper, top=20.0mm, bottom=20.0mm, left=20.0mm, right=20.0mm,
includefoot]{geometry}

\usepackage{ragged2e} % центрирование текста

\usepackage{tikz} % картинки в tikz
\usepackage{microtype} % свешивание пунктуации

\usepackage{floatrow} % для выравнивания рисунка и подписи
\usepackage{caption} % для пустых подписей

\usepackage{array} % для столбцов фиксированной ширины

\usepackage{indentfirst} % отступ в первом параграфе

\usepackage{sectsty} % для центрирования названий частей
\allsectionsfont{\centering}

\usepackage{amsmath, amsfonts} % куча стандартных математических плюшек

\usepackage{comment} % для комментариев

\usepackage{multicol} % текст в несколько колонок

\usepackage{lastpage} % чтобы узнать номер последней страницы

\usepackage{enumitem} % дополнительные плюшки для списков
%  например \begin{enumerate}[resume] позволяет продолжить нумерацию в новом списке

\usepackage{url} % для вставки интернет-ссылок

\usepackage{fontspec}
\usepackage{polyglossia}

\setmainlanguage{russian}
\setotherlanguages{english}

% download "Linux Libertine" fonts:
% http://www.linuxlibertine.org/index.php?id=91&L=1
\setmainfont{Linux Libertine O} % or Helvetica, Arial, Cambria
% why do we need \newfontfamily:
% http://tex.stackexchange.com/questions/91507/
\newfontfamily{\cyrillicfonttt}{Linux Libertine O}

\AddEnumerateCounter{\asbuk}{\russian@alph}{щ} % для списков с русскими буквами
\setlist[enumerate, 2]{label=\asbuk*),ref=\asbuk*}

\DeclareMathOperator{\Var}{Var}
\DeclareMathOperator{\E}{\mathbb{E}}

\let\P\relax
\DeclareMathOperator{\P}{\mathbb{P}}
\def\cN{\mathcal{N}}

\usepackage{fancyhdr} % весёлые колонтитулы
\pagestyle{fancy}
\lhead{}
\chead{}
\rhead{}
\lfoot{}
\cfoot{}
\rfoot{}
\renewcommand{\headrulewidth}{0.4pt}
\renewcommand{\footrulewidth}{0.4pt}

\providecommand{\tightlist}{%
  \setlength{\itemsep}{0pt}\setlength{\parskip}{0pt}}

\begin{document}


\begin{center}
{\small Федеральное государственное автономное образовательное учреждение\\ 
высшего профессионального образования «Национальный исследовательский\\ 
университет «Высшая школа экономики».}
\end{center}

\vspace{0.4cm}

\begin{center}
\textbf{Рецензия на выпускную квалификационную работу}
\end{center}

\vspace{0.4cm}

Студент ОП «Экономика»: Жуйков Дмитрий Игоревич

\vspace{0.4cm}

Научный руководитель: Малаховская Оксана Анатольевна

\vspace{0.4cm}

Тема: Макроэкономическое прогнозирование с использованием данных различной
частотности

\vspace{0.4cm}

В своей работе Дмитрий сравнивает прогнозную силу MIDAS и двумерных VAR
моделей для российского ВВП в краткосрочном периоде.

Дмитрий разобрался с литературой по MIDAS моделям и проводит аналогичные
построения на российских данных. Ясно виден большой объём проделанной
Дмитрием работы. Есть чётко сформулированные ценные результаты: выявлено
преимущество VAR моделей для прогнозирования будущих значений ВВП и
небольшое преимущество MIDAS моделей для прогнозирования текущих
значений ВВП (nowcasting). Для оценивания моделей Дмитрий использовал
статистический пакет R.

О недостатках. Местами работу можно сократить. Фразы типа «применены как
общенаучные, так и специфические методы экономической науки» не несут
читателю никакой информации. Модель MIDAS на странице 17 Дмитрий
описывает небрежно. Двойной индекс \(k\) в формуле, отсутствие описания
смысла величины \(T_m^x\).

Процедура сезонной корректировки описана поверхностно. Все ли ряды
корректировались? С какими параметрами запускалась процедура X13-ARIMA?
Графики рядов на рисунках 1-5 --- это прогнозы скорректированные на
сезонность? Или сделано преобразование прогнозов с очищенной сезонностью
обратно к исходным данным? Кстати, названия рисунков неудачные.
Например, рисунок 1 называется ``Инфляция''. Что на нём изображено?
Правильно, на нём изображён темп роста ВВП.

В работе не хватает грамотного графического представления результатов.
Во-первых, можно попытаться изображать не сами ряды и прогнозы, а ошибки
прогнозов. Во-вторых, в дополнение к таблицам отношений RMSE можно
добавить сгруппированные по моделям и горизонтам оценённые плотности
ошибок прогнозов. В третьих, не хватает графического представления
исходных рядов до периода оценивания.

Настоятельно советую Дмитрию выложить код работы с комментариями в
открытый доступ, для воспроизводимости результатов.

Работа Дмитрия заслуживает отличной оценки.

\vspace{0.4cm}

21 мая 2017

Демешев Борис Борисович



\end{document}

\documentclass[a4paper, 12pt]{article}


\usepackage{mathrsfs}

\usepackage{amscd}
\usepackage[paper=a4paper, top=20.0mm, bottom=20.0mm, left=20.0mm, right=20.0mm,
includefoot]{geometry}

\usepackage{ragged2e} % центрирование текста

\usepackage{tikz} % картинки в tikz
\usepackage{microtype} % свешивание пунктуации

\usepackage{floatrow} % для выравнивания рисунка и подписи
\usepackage{caption} % для пустых подписей

\usepackage{array} % для столбцов фиксированной ширины

\usepackage{indentfirst} % отступ в первом параграфе

\usepackage{sectsty} % для центрирования названий частей
\allsectionsfont{\centering}

\usepackage{amsmath, amsfonts} % куча стандартных математических плюшек

\usepackage{comment} % для комментариев

\usepackage{multicol} % текст в несколько колонок

\usepackage{lastpage} % чтобы узнать номер последней страницы

\usepackage{enumitem} % дополнительные плюшки для списков
%  например \begin{enumerate}[resume] позволяет продолжить нумерацию в новом списке

\usepackage{url} % для вставки интернет-ссылок

\usepackage{fontspec}
\usepackage{polyglossia}

\setmainlanguage{russian}
\setotherlanguages{english}

% download "Linux Libertine" fonts:
% http://www.linuxlibertine.org/index.php?id=91&L=1
\setmainfont{Linux Libertine O} % or Helvetica, Arial, Cambria
% why do we need \newfontfamily:
% http://tex.stackexchange.com/questions/91507/
\newfontfamily{\cyrillicfonttt}{Linux Libertine O}

\AddEnumerateCounter{\asbuk}{\russian@alph}{щ} % для списков с русскими буквами
\setlist[enumerate, 2]{label=\asbuk*),ref=\asbuk*}

\DeclareMathOperator{\Var}{Var}
\DeclareMathOperator{\E}{\mathbb{E}}

\let\P\relax
\DeclareMathOperator{\P}{\mathbb{P}}
\def\cN{\mathcal{N}}

\usepackage{fancyhdr} % весёлые колонтитулы
\pagestyle{fancy}
\lhead{}
\chead{}
\rhead{}
\lfoot{}
\cfoot{}
\rfoot{}
\renewcommand{\headrulewidth}{0.4pt}
\renewcommand{\footrulewidth}{0.4pt}

\providecommand{\tightlist}{%
  \setlength{\itemsep}{0pt}\setlength{\parskip}{0pt}}

\begin{document}


\begin{center}
{\small Федеральное государственное автономное образовательное учреждение\\ 
высшего профессионального образования «Национальный исследовательский\\ 
университет «Высшая школа экономики».}
\end{center}

\vspace{0.4cm}

\begin{center}
\textbf{Отзыв на выпускную квалификационную работу}
\end{center}

\vspace{0.4cm}

Студент ОП «Экономика»: Звонка Георгий Дмитриевич

\vspace{0.4cm}

Научный руководитель: Демешев Борис Борисович

\vspace{0.4cm}

Тема: Сбор и анализ данных с российских анонимных торговых площадок

\vspace{1cm}

Работа Геогрия посвящена сбору и статистическому анализу данных
интернет-торговли наркотиками. Георгий рассматривает две лидирующие по
объёму сделок площадки --- RAMP и HYDRA.

В работе Георгию удалось:

\begin{enumerate}
\def\labelenumi{\arabic{enumi}.}
\item
  Разработать и успешно реализовать методологию анонимного сбора
  информации по количеству сделок и их медианной цене.
\item
  Впервые получить объективные количественные оценки объёма
  интернет-торговли наркотиками по городам России. И это не экспертные
  оценки, а результаты обработки данных реальных сделок!
\item
  Построить индекс торговой активности анонимных площадок.
\item
  Проверить гипотезу о влиянии курса биткойна на индекс торговой
  активности.
\end{enumerate}

При выполнении работы Георгию пришлось столкнуться со множеством
сложностей. Достаточно отметить, что составление одного слепка состояния
всех площадок занимает порядка 10 часов. И эти гигабайтные слепки нужно
связывать между собой.

В силу технических сложностей парсинга торговых площадок сети TOR
полученный временной ряд довольно короткий и содержит отдельные
пропуски. Георгий разобрался с фильтром Калмана и успешно использовал
его для заполнения пропусков в данных. К сожалению, никакие
статистические методы не могут удлиннить имеющийся ряд данных.

Работать Георгий начал интенсивно ещё до Нового года. Для своих задач
успешно освоил и парсинг html-страниц с помощью python, и статистический
анализ в R.

Работу нельзя назвать идеально оформленной, а изложение местами
затянуто.

Было бы очень интересно продолжить получение ряда данных для проведения
более содержательного анализа, благо разработанная методология позволяет
это сделать без существенных затрат труда.

Работа Георгия заслуживает отличной оценки.

\vspace{1cm}

15 мая 2017

Демешев Борис Борисович



\end{document}

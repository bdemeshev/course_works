\documentclass[a4paper, 12pt]{article}


\usepackage{mathrsfs}

\usepackage{amscd}
\usepackage[paper=a4paper, top=20.0mm, bottom=20.0mm, left=20.0mm, right=20.0mm,
includefoot]{geometry}

\usepackage{ragged2e} % центрирование текста

\usepackage{tikz} % картинки в tikz
\usepackage{microtype} % свешивание пунктуации

\usepackage{floatrow} % для выравнивания рисунка и подписи
\usepackage{caption} % для пустых подписей

\usepackage{array} % для столбцов фиксированной ширины

\usepackage{indentfirst} % отступ в первом параграфе

\usepackage{sectsty} % для центрирования названий частей
\allsectionsfont{\centering}

\usepackage{amsmath, amsfonts} % куча стандартных математических плюшек

\usepackage{comment} % для комментариев

\usepackage{multicol} % текст в несколько колонок

\usepackage{lastpage} % чтобы узнать номер последней страницы

\usepackage{enumitem} % дополнительные плюшки для списков
%  например \begin{enumerate}[resume] позволяет продолжить нумерацию в новом списке

\usepackage{url} % для вставки интернет-ссылок

\usepackage{fontspec}
\usepackage{polyglossia}

\setmainlanguage{russian}
\setotherlanguages{english}

% download "Linux Libertine" fonts:
% http://www.linuxlibertine.org/index.php?id=91&L=1
\setmainfont{Linux Libertine O} % or Helvetica, Arial, Cambria
% why do we need \newfontfamily:
% http://tex.stackexchange.com/questions/91507/
\newfontfamily{\cyrillicfonttt}{Linux Libertine O}

\AddEnumerateCounter{\asbuk}{\russian@alph}{щ} % для списков с русскими буквами
\setlist[enumerate, 2]{label=\asbuk*),ref=\asbuk*}

\DeclareMathOperator{\Var}{Var}
\DeclareMathOperator{\E}{\mathbb{E}}

\let\P\relax
\DeclareMathOperator{\P}{\mathbb{P}}
\def\cN{\mathcal{N}}

\usepackage{fancyhdr} % весёлые колонтитулы
\pagestyle{fancy}
\lhead{}
\chead{}
\rhead{}
\lfoot{}
\cfoot{}
\rfoot{}
\renewcommand{\headrulewidth}{0.4pt}
\renewcommand{\footrulewidth}{0.4pt}

\providecommand{\tightlist}{%
  \setlength{\itemsep}{0pt}\setlength{\parskip}{0pt}}

\begin{document}


\begin{center}
{\small Федеральное государственное автономное образовательное учреждение\\ 
высшего профессионального образования «Национальный исследовательский\\ 
университет «Высшая школа экономики».}
\end{center}

\vspace{0.4cm}

\begin{center}
\textbf{Рецензия на выпускную квалификационную работу}
\end{center}

\vspace{0.4cm}

Студент ОП «Экономика»: Гимадиев Азат

\vspace{0.4cm}

Научный руководитель: Демешев Борис Борисович

\vspace{0.4cm}

Тема: Прогнозирование стоимости опционов с помощью моделей краткосрочных
процентных ставок

\vspace{0.4cm}

Работа Азата посвящена прогнозированию цены опционов. В названии
заявлено использование моделей краткосрочных процентных ставок, но до
этих моделей, к сожалению, в работе дело не доходит. Азат прогнозирует
цену опциона в рамках наивных предположений о динамике цены акции.

Плюсы работы. С одной стороны Азат разобрался в базовых теоретических
моделях оценивания опционов. Несмотря на некоторые мелкие неточности в
описании модели, в целом видно, что Азат разобрался с моделью
Блэка-Шоулса. С другой стороны, Азат работает с реальными данными и
демонстрирует умение обработать сырые данные и реализовать алгоритм
прогнозирования на языке python.

Минусы работы. Работа состоит из отдельных кусков: блок теории и почти
не связанный практический блок с прогнозами для четырех опционов на
разные акции. Изложение трудно назвать аккуратным, но грубых ошибок нет.
Также остались нереализованными прогнозы с моделированием динамики
процентных ставок.

Трудно оценивать работу студента другого факультета, поскольку нет
возможности выбрать точку отсчёта. По меркам факультета экономики работа
Азата заслуживает удовлетворительную или хорошую оценку.

\vspace{0.4cm}

16 июня 2017

Демешев Борис Борисович



\end{document}

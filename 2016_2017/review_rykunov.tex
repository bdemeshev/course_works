\documentclass[a4paper, 12pt]{article}


\usepackage{mathrsfs}

\usepackage{amscd}
\usepackage[paper=a4paper, top=20.0mm, bottom=20.0mm, left=20.0mm, right=20.0mm,
includefoot]{geometry}

\usepackage{ragged2e} % центрирование текста

\usepackage{tikz} % картинки в tikz
\usepackage{microtype} % свешивание пунктуации

\usepackage{floatrow} % для выравнивания рисунка и подписи
\usepackage{caption} % для пустых подписей

\usepackage{array} % для столбцов фиксированной ширины

\usepackage{indentfirst} % отступ в первом параграфе

\usepackage{sectsty} % для центрирования названий частей
\allsectionsfont{\centering}

\usepackage{amsmath, amsfonts} % куча стандартных математических плюшек

\usepackage{comment} % для комментариев

\usepackage{multicol} % текст в несколько колонок

\usepackage{lastpage} % чтобы узнать номер последней страницы

\usepackage{enumitem} % дополнительные плюшки для списков
%  например \begin{enumerate}[resume] позволяет продолжить нумерацию в новом списке

\usepackage{url} % для вставки интернет-ссылок

\usepackage{fontspec}
\usepackage{polyglossia}

\setmainlanguage{russian}
\setotherlanguages{english}

% download "Linux Libertine" fonts:
% http://www.linuxlibertine.org/index.php?id=91&L=1
\setmainfont{Linux Libertine O} % or Helvetica, Arial, Cambria
% why do we need \newfontfamily:
% http://tex.stackexchange.com/questions/91507/
\newfontfamily{\cyrillicfonttt}{Linux Libertine O}

\AddEnumerateCounter{\asbuk}{\russian@alph}{щ} % для списков с русскими буквами
\setlist[enumerate, 2]{label=\asbuk*),ref=\asbuk*}

\DeclareMathOperator{\Var}{Var}
\DeclareMathOperator{\E}{\mathbb{E}}

\let\P\relax
\DeclareMathOperator{\P}{\mathbb{P}}
\def\cN{\mathcal{N}}

\usepackage{fancyhdr} % весёлые колонтитулы
\pagestyle{fancy}
\lhead{}
\chead{}
\rhead{}
\lfoot{}
\cfoot{}
\rfoot{}
\renewcommand{\headrulewidth}{0.4pt}
\renewcommand{\footrulewidth}{0.4pt}

\providecommand{\tightlist}{%
  \setlength{\itemsep}{0pt}\setlength{\parskip}{0pt}}

\begin{document}


\begin{center}
{\small Федеральное государственное автономное образовательное учреждение\\ 
высшего профессионального образования «Национальный исследовательский\\ 
университет «Высшая школа экономики».}
\end{center}

\vspace{0.4cm}

\begin{center}
\textbf{Рецензия на выпускную квалификационную работу}
\end{center}

\vspace{0.4cm}

Студент ОП «Экономика»: Дмитрий Рыкунов

\vspace{0.4cm}

Научный руководитель: Станкевич Иван Павлович

\vspace{0.4cm}

Тема: Численные методы нахождения множественных равновесий в теоретико-игровых
моделях

\vspace{1cm}

Работа Дмитрия посвящена созданию алгоритма, способного численно
находить равновесия в игровых моделях. Дмитрий реализует алгоритм на
питоне и проверяет его работу на нескольких играх. Алгоритм успешно
справляется с поиском равновесий в рассмотренных играх.

Введение работы содержит очень много воды. Работа про алгоритм поиска
равновесий, а во введении «С самых древних эпох, от доисторических
племён, совместной охоты, великих войн, завоеваний и рабства\ldots{}»
Есть во введении и путешествия во времени: «Начав с макроэкономических
теорий\ldots{} (1993) \ldots{} (1936), экономическая теория продолжила
микроэкономикой \ldots{} (1890)».

В обзоре литературы Дмитрий пишет, что нахождение всех равновесий Нэша
является, «возможно», NP-трудной задачей. А это давно известный факт.
Например, уже в симметричной игре двух игроков, даже ответ на вопрос,
единственно ли равновесие или нет, является NP-трудной задачей.
Доказательство можно найти в Gilboa, Zemel «Nash and correlated
equilibria: Some complexity considerations». Есть и другие мелкие
небрежности. Например, Дмитрий пишет, что функция полезности может быть
и случайной величиной, однако не ясно, как это соответствует формальной
записи на странице 17. В формулировке задаче агента на стр. 18 пропущен
минус перед индексом \(i\). О смысле таблицы 3, названной рисунком 3,
читателю надо догадываться самому. И так далее.

Созданный код Дмитрий выложил в открытый доступ на гитхаб-репозиторий,
что хорошо, однако ссылок на него в работе не привёл. Код написан на
очень высоком для бакалавра уровне, однако документирован слабо, что
затрудняет использование алгоритма другими исследователями.

Наиболее существенным недостатком работы является отсутствие сравнения
придуманного алгоритма с другими алгоритмами.

Работа Дмитрия написана на английском языке. С одной стороны, в
современных академических реалиях это плюс, с другой стороны,
шероховатости языка бросаются в глаза. Их так много, что приятней читать
было бы на русском. Я бы посоветовал Дмитрию вычитать текст с носителем
языка или хотя бы разбить все длинные предложения на более простые, а
также воспользоваться бесплатными сервисами типа paperrater или
grammarly.

Работа Дмитрия заслуживает отличной оценки.

\vspace{1cm}

21 мая 2017

Демешев Борис Борисович



\end{document}

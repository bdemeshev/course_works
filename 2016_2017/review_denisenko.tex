\documentclass[a4paper, 12pt]{article}


\usepackage{mathrsfs}

\usepackage{amscd}
\usepackage[paper=a4paper, top=20.0mm, bottom=20.0mm, left=20.0mm, right=20.0mm,
includefoot]{geometry}

\usepackage{ragged2e} % центрирование текста

\usepackage{tikz} % картинки в tikz
\usepackage{microtype} % свешивание пунктуации

\usepackage{floatrow} % для выравнивания рисунка и подписи
\usepackage{caption} % для пустых подписей

\usepackage{array} % для столбцов фиксированной ширины

\usepackage{indentfirst} % отступ в первом параграфе

\usepackage{sectsty} % для центрирования названий частей
\allsectionsfont{\centering}

\usepackage{amsmath, amsfonts} % куча стандартных математических плюшек

\usepackage{comment} % для комментариев

\usepackage{multicol} % текст в несколько колонок

\usepackage{lastpage} % чтобы узнать номер последней страницы

\usepackage{enumitem} % дополнительные плюшки для списков
%  например \begin{enumerate}[resume] позволяет продолжить нумерацию в новом списке

\usepackage{url} % для вставки интернет-ссылок

\usepackage{fontspec}
\usepackage{polyglossia}

\setmainlanguage{russian}
\setotherlanguages{english}

% download "Linux Libertine" fonts:
% http://www.linuxlibertine.org/index.php?id=91&L=1
\setmainfont{Linux Libertine O} % or Helvetica, Arial, Cambria
% why do we need \newfontfamily:
% http://tex.stackexchange.com/questions/91507/
\newfontfamily{\cyrillicfonttt}{Linux Libertine O}

\AddEnumerateCounter{\asbuk}{\russian@alph}{щ} % для списков с русскими буквами
\setlist[enumerate, 2]{label=\asbuk*),ref=\asbuk*}

\DeclareMathOperator{\Var}{Var}
\DeclareMathOperator{\E}{\mathbb{E}}

\let\P\relax
\DeclareMathOperator{\P}{\mathbb{P}}
\def\cN{\mathcal{N}}

\usepackage{fancyhdr} % весёлые колонтитулы
\pagestyle{fancy}
\lhead{}
\chead{}
\rhead{}
\lfoot{}
\cfoot{}
\rfoot{}
\renewcommand{\headrulewidth}{0.4pt}
\renewcommand{\footrulewidth}{0.4pt}

\providecommand{\tightlist}{%
  \setlength{\itemsep}{0pt}\setlength{\parskip}{0pt}}

\begin{document}


\begin{center}
{\small Федеральное государственное автономное образовательное учреждение\\ 
высшего профессионального образования «Национальный исследовательский\\ 
университет «Высшая школа экономики».}
\end{center}

\vspace{0.4cm}

\begin{center}
\textbf{Рецензия на выпускную квалификационную работу}
\end{center}

\vspace{0.4cm}

Студент ОП «Экономика»: Денисенко Анна Андреевна

\vspace{0.4cm}

Научный руководитель: Сонин Кирилл Исаакович

\vspace{0.4cm}

Тема: Путь к демократии: эмиграция, подчинение и протест

\vspace{1cm}

В своей работе Анна описывает теоретическую игровую модель
взаимодействия трёх игроков: правителя, советника и населения. Описав
модель формально, Анна аккуратно изучает свойства модели и находит
равновесия.

Несомненным плюсом работы является то, что Анна находит решение модели
аналитически. При этом Анна демонстрирует уверенное владение игровыми
концепциями. Все переходы между формулами я не проверял, но там, где
проверил, никаких ошибок не нашёл. Аналитическое решение модели на более
чем 20 страницах занимает основную часть работы.

Рисунок 2 приведён без ссылок на оригинал, да и вызывает естественный
вопрос: почему все режимы не скатываются в анократию? Судя по рисунку: в
автократии положительных изменений больше, чем отрицательных, а в
демократии --- наоборот, положительных нет, а отрицательные есть. При
этом во введении Анна пишет, что анократия --- «хрупкое равновесие»,
хотя такая динамика --- скорее признак устойчивого равновесия.

В описании модели лучше было сразу добавить дерево. Без дерева игры
воспринимать две с лишним страницы тексто тяжело. Пусть общее дерево
игры со всеми стратегиями и было бы громоздким, с ним изложение было бы
яснее. Анна же строит деревья только в процессе решения и только
кусками.

Существенным недостатком работы является только теоретическое
рассмотрение вопроса. Введение содержит примеры реальных политических
конфликтов. Какое отношение имеют эти примеры к рассматриваемому дереву
игры, не понятно. Анна не делает попыток калибровки модели: предпосылки
модели не сопоставляются с реальными конфликтами. Нет и попыток что-либо
спрогнозировать с помощью этой модели. Поэтому мне очень понравилось,
что советник иногда называется визирем. Правителя нужно было назвать
султаном и ввести в игру Шахерезаду. Это модель-сказка :)

Работа очень красиво оформлена. Разве что после точек при сокращении
имён в техе правильнее ставить неразрывный пробел тильдой. Пожалуй, это
наиболее приятно оформленная работа из десятка ВКР прочитанных мной в
этом году.

Работа Анны заслуживает отличной оценки.

\vspace{1cm}

21 мая 2017

Демешев Борис Борисович



\end{document}

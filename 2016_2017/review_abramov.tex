\documentclass[a4paper, 12pt]{article}


\usepackage{mathrsfs}

\usepackage{amscd}
\usepackage[paper=a4paper, top=20.0mm, bottom=20.0mm, left=20.0mm, right=20.0mm,
includefoot]{geometry}

\usepackage{ragged2e} % центрирование текста

\usepackage{tikz} % картинки в tikz
\usepackage{microtype} % свешивание пунктуации

\usepackage{floatrow} % для выравнивания рисунка и подписи
\usepackage{caption} % для пустых подписей

\usepackage{array} % для столбцов фиксированной ширины

\usepackage{indentfirst} % отступ в первом параграфе

\usepackage{sectsty} % для центрирования названий частей
\allsectionsfont{\centering}

\usepackage{amsmath, amsfonts} % куча стандартных математических плюшек

\usepackage{comment} % для комментариев

\usepackage{multicol} % текст в несколько колонок

\usepackage{lastpage} % чтобы узнать номер последней страницы

\usepackage{enumitem} % дополнительные плюшки для списков
%  например \begin{enumerate}[resume] позволяет продолжить нумерацию в новом списке

\usepackage{url} % для вставки интернет-ссылок

\usepackage{fontspec}
\usepackage{polyglossia}

\setmainlanguage{russian}
\setotherlanguages{english}

% download "Linux Libertine" fonts:
% http://www.linuxlibertine.org/index.php?id=91&L=1
\setmainfont{Linux Libertine O} % or Helvetica, Arial, Cambria
% why do we need \newfontfamily:
% http://tex.stackexchange.com/questions/91507/
\newfontfamily{\cyrillicfonttt}{Linux Libertine O}

\AddEnumerateCounter{\asbuk}{\russian@alph}{щ} % для списков с русскими буквами
\setlist[enumerate, 2]{label=\asbuk*),ref=\asbuk*}

\DeclareMathOperator{\Var}{Var}
\DeclareMathOperator{\E}{\mathbb{E}}

\let\P\relax
\DeclareMathOperator{\P}{\mathbb{P}}
\def\cN{\mathcal{N}}

\usepackage{fancyhdr} % весёлые колонтитулы
\pagestyle{fancy}
\lhead{}
\chead{}
\rhead{}
\lfoot{}
\cfoot{}
\rfoot{}
\renewcommand{\headrulewidth}{0.4pt}
\renewcommand{\footrulewidth}{0.4pt}

\providecommand{\tightlist}{%
  \setlength{\itemsep}{0pt}\setlength{\parskip}{0pt}}

\begin{document}


\begin{center}
{\small Федеральное государственное автономное образовательное учреждение\\ 
высшего профессионального образования «Национальный исследовательский\\ 
университет «Высшая школа экономики».}
\end{center}

\vspace{0.4cm}

\begin{center}
\textbf{Рецензия на выпускную квалификационную работу}
\end{center}

\vspace{0.4cm}

Студент ОП «Экономика»: Абрамов Владислав Викторович

\vspace{0.4cm}

Научный руководитель: Малаховская Оксана Анатольевна

\vspace{0.4cm}

Тема: Расчет фискального мультипликатора для российской экономики в рамках
VAR-модели со знаковыми ограничениями

\vspace{0.4cm}

Работа Владислава посвящена оценке влияния шоков государственного
потребления и государственных расходов на выпуск по российским данным. Я
скептически отношусь к осмысленности оценки влияния ненаблюдаемых
величин на наблюдаемые с помощью функций импульсного отклика и
дальнейшей их интерпретации. Однако здесь Владислав следует довольно
популярной в литературе методологии.

Работа оставляет при прочтении ощущение вопиющей небрежности.

В списке литературы дважды указана работа Friedman, Benjamin M. (1978),
Crowding Out Or Crowding In? Также дважды в списке литературы указана
работа Buiter, (1977), ``Crowding out' and the effectiveness of fiscal
policy''. Причём один раз работа Buiter указана с неверным годом (1973)
и на эту неверную дату есть ссылка в тексте на странице 7. На странице
10 Владислав цитирует ``Canova and Pappa (2003, 2008)'', а в списке
литературы есть только Canova and Pappa (2006). На странице 10 Владислав
цитирует Canova, Pina (2005), а в списке литературы есть только Canova,
Pina (1998). На странице 27 работа Mountford, Uhlig (2009) цитируется
три раза, два из которых с опечатками. По тексту работы Владислава есть
Ilzetzki (2009), (2011) и (2013), а в списке литературы --- только
Ilzetzki (2013). И так далее.

Бездоказательно Владислав высказывает утверждение о бесполезности
панельных VAR (страница 13). На мой взгляд, именно учет информации о
похожих странах может помочь улучшить прогнозы VAR для отдельной страны.
Двумя страницами позже разгрома панельных VAR Владислав называет работу
по панельным VAR «знаковой», и пишет, что «эмпирические свидетельства
подтверждаются теоретическими аргументами».

Надеюсь, что фраза «\textbf{по мнению авторов}», написанная Владиславом
на странице 19 о своей работе, является простой опиской.

Инфляция в таблице 1 явно измеряется в долях, а Владислав утверждает,
что в процентах. Владислав крайне скупо описывает, что де-факто было
сделано с данными. По каким критериям были отобраны 8 переменных?

Как конкретно удалялась сезонность? «Для нивелирования
сезонности\ldots{} мы предположили, что ряды носят мультипликативный
характер» --- это не годится для описания метода. Более того, Владислав
пишет, что брал \textbf{квартальные данные}. Однако график частных
корреляций государственных доходов на странице 45 однозначно указывает
на \textbf{месячные данные}. График PACF даёт высокие по модулю значения
на 1, 12 и 13 лагах. Это объяснимо только для месячных данных, где
текущий месяц зависит от прошлого месяца и от аналогичного месяца
прошлого года. На мысль об использовании \textbf{месячных данных}
наталкивает и график на странице 39. На графике государственных доходов
легко видна годовая цикличность. А если перевести взгляд на график выше,
то видно что в один цикл (один год) явно умещается больше четырёх
изломов верхнего графика. С другой стороны, количество изломов на
графиках на странице 40 указывает на \textbf{квартальные данные}.
Возможно, что ряд по государственным доходам взят месячный?

Как конкретно производилась идентификация, из работы тоже не очень ясно.
Алгоритм на странице 24 --- это плохой перевод не самого лучшего
изложения алгоритма из документации пакета \texttt{VARsignR}. Здесь
уместно сказать и слова в защиту Владислава. Владислав использует очень
сложный алгоритм, существенно выходящий за рамки бакалаврской и даже
магистерской программы. У меня создалось ощущение, что, с одной стороны,
понимание алгоритма у Владислава поверхностное, а с другой стороны, что
во всей ВШЭ найдётся не больше пяти человек, понимающих алгоритм лучше
Владислава.

При написании ВКР Владислав использовал готовый пакет \texttt{VARsignR}
для статистической среды R. Также в R построены требуемые графики.

С учётом сложности алгоритма идентификации шоков работа Владислава
заслуживает удовлетворительной оценки.

\vspace{0.4cm}

21 мая 2017

Демешев Борис Борисович



\end{document}
